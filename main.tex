\documentclass[10pt, a4paper, twoside]{article}
\usepackage[utf8]{inputenc}
\usepackage{newtxtext,newtxmath} % Times New Roman font
\usepackage[left=2.5cm, right=2.5cm, top=2.5cm, bottom=2.5cm, headheight=15pt]{geometry}
\usepackage{titlesec} % For customizing section titles
\usepackage{enumitem} % Enables bullet points
\usepackage{lipsum} % Add dummy text
\usepackage{hyperref} % Links for references
\usepackage{parskip} % Set spacing between paragraphs
\usepackage{fancyhdr} % Format headers
\usepackage{etoolbox} % Format bibliography header
\usepackage{physics} % Used to write differential equations more easily
% \usepackage{siunitx} % Used to write SI base units

\setlength{\parskip}{10pt}

\titleformat{\section}{\normalfont\fontsize{10}{13}\uppercase}{\thesection.}{1em}{}
\titleformat{\subsection}{\normalfont\fontsize{10}{13}\bfseries}{\thesubsection.}{1em}{}
\titleformat{\subsubsection}{\normalfont\fontsize{10}{13}\itshape}{\thesubsubsection.}{1em}{}

% Redefining \section* command for centered section title
\titleformat{name=\section,numberless}{\normalfont\fontsize{10}{13}\bfseries\centering\uppercase}{}{0pt}{}

% Ensure references title has the same format as \section*
\patchcmd{\thebibliography}{\section*{\refname}}{\section*{REFERENCES}}{}{}

%  Create headers
\fancyhf{} % clear all header and footer fields
\renewcommand{\headrulewidth}{0pt}
\fancyhead[CE]{\fontsize{8}{10}\selectfont \textbf{IAEA-CN-316/2143}}
\fancyhead[CO]{\fontsize{8}{10}\selectfont \textbf{PROKOPYSZYN et al.}}
\pagestyle{fancy}

\begin{document}
\begin{flushleft}
\fontsize{12}{14}\selectfont \textbf{CONFINEMENT OF FUSION ALPHA-PARTICLES AND ALFV\'EN EIGENMODE STABILITY IN STEP}
% \fontsize{12}{14}\selectfont \textbf{\textit{Subtitle if needed in Times New Roman 12 point bold italic, sentence case}}

\fontsize{10}{13}\selectfont
A.P.K. PROKOPYSZYN, K.G. MCCLEMENTS, H.J.C. OLIVER, M. FITZGERALD, D.A. RYAN, G. XIA
United Kingdom Atomic Energy Authority \\
Culham Centre for Fusion Energy, Culham Science Centre, Abingdon, Oxfordshire OX14 3DB, United Kingdom \\
Email: alex.prokopysyzn@ukaea.uk

\end{flushleft}

% Abstract section
\begin{flushleft}
\textbf{Abstract}
\end{flushleft}

\setlength{\parindent}{1cm}
\fontsize{9}{12pt}\selectfont

The Spherical Tokamak for Energy Production (STEP) programme is focused on constructing a state-of-the-art fusion power plant prototype that will generate approximately 1.6-1.7 GW of deuterium-tritium fusion energy. To achieve this, the $\alpha$-particles generated through fusion must be adequately confined to maintain the necessary high temperature in the core of the plasma and to protect the wall from too much damage. Microwaves are used for both external heating and current drive, making $\alpha$-particles the only significant fast-ion species. The purpose of this research is to model the confinement of $\alpha$-particles and the toroidal Alfvén eigenmodes (TAEs) driven by these particles in a variety of plasma scenarios to help determine the best configuration. The scenarios examined here have been identified by the STEP team as potential flat-top operating configurations. We use LOCUST (Lorentz Orbit Code for Use in Stellerators and Tokamaks) to model the $\alpha$-particle confinement and heat-load distribution on the wall, and HALO (HAgis LOcust) to model the TAEs. The results indicate that adequate confinement can be achieved in some of the scenarios, but caution must be taken. For instance, slight modifications to the RMP coils can reduce the confinement by a factor of ten.

% Abstract and title over, format text for the main body
\setlength{\parindent}{0pt}
\fontsize{10}{13}\selectfont

\section{Introduction}

The United Kingdom Atomic Energy Authority (UKAEA) is pioneering efforts to develop a compact prototype of a fusion energy power plant and establish a commercial pathway for fusion energy \cite{nuttall2020, meyer2023}. A crucial factor in the efficiency of a Deuterium-Tritium burning fusion reactor is the confinement of fusion-born $\alpha$-particles. These particles hold the energy that fuels further fusion reactions. Furthermore, it is imperative to minimise particle losses to prevent potential damage to the reactor walls.

Ideally, $\alpha$-particles should slow down through Coulomb collisions with thermal electrons and ions, thus transferring their energy to facilitate further fusion reactions. However, under realistic conditions, high-frequency instabilities may divert $\alpha$-ion energy away from the reactor core \cite{belova2015}. Toroidal Alfv\'en eigenmodes (TAEs) exemplify the type of plasma instability that can significantly disrupt plasma confinement. The present research investigates plasma stability against TAEs. The objective is to inform the design process and reduce the probability of TAE instabilities in the system.

External heating and current drive for the plasma in STEP during flat-top will be provided entirely by microwaves. This heating strategy will utilize a combination of electron cyclotron current drive and electron Bernstein waves. Detailed information on the proposed heating and current drive systems for STEP is provided in \cite{freethy2023}.
This approach has significant implications, as it means that the only substantial population of fast ions will be composed of $\alpha$-particles, which are a product of the fusion reaction itself. Unlike in other tokamak devices such as JET and ITER, STEP will not generate fast ions via neutral beam injectors or ion cyclotron resonance heating systems.
In this context, `fast ions' are defined as ions with speeds that are significantly greater than those of the background plasma, making them `suprathermal'. For example, in the STEP design, the plasma at the core is expected to reach a temperature of approximately 20 keV (see, e.g. \cite{meyer2023,mitchell2023}), whereas the fusion-born $\alpha$-particles will carry an energy of 3.5 MeV, much higher than the background plasma.

For safety and economic reasons, the plasma-facing components (PFCs) must be able to effectively contain the plasma and be highly durable to prevent frequent damage. During operation, the PFCs will be exposed to heat from various sources, even if disruptions and instabilities are successfully avoided. The wall is exposed to radiation from the plasma, neutron radiation emitted during fusion, thermally charged particles that escape, runaway electrons, erosion due to sputtering, and fast $\alpha$ -particles that escape. This research focuses on the contribution of $\alpha$-particles. See, for example, \cite{zsolt2023} for an overview of the current modelling of the total heat load on the PFCs and \cite{fil2023} for an overview of the modeling of runaway electrons for STEP. Our objective is to maintain the heat load on the wall to be less than approximately $1\, \text{MW/m}^2$ on the first wall, about $5\, \text{MW/m}^2$ on the nose structures, and $10\, \text{MW/m}^2$ on the divertors. It is important to note that our wall, while incorporating limiters, remains smooth and axisymmetric, although the final design will be comparatively less smooth. Consequently, we must achieve lower heat loads to compensate for shadowing, which we estimate could potentially increase the peak heat load.

Owing to the high speed of the $\alpha$-particles, which results in large Larmor radii, their confinement can easily be compromised due to deviations in the background magnetic field from axisymmetric to a three-dimensional structure.  In this study, we model $\alpha$-particles in a field that is predominantly axisymmetric, except for two three-dimensional contributions. The first nonaxisymmetric contribution stems from the ripple magnetic field introduced by the use of a finite number of toroidal field (TF) coils. The second arises from edge-localized mode (ELM) \cite{zohm1996} mitigation coils, which generate a three-dimensional field with the aim of suppressing edge-localized modes.

STEP aims to operate in high-confinement mode (H-mode) \cite{wagner1982}, as a result, type-I ELMs may pose a significant problem. To address this issue, STEP will employ ELM mitigation coils to generate resonant magnetic perturbations to suppress ELMs. This approach has been successfully demonstrated experimentally in previous studies, such as \cite{suttrop2018}. However, the field created by the ELM mitigation coils disrupts the axisymmetry of the system and consequently affects the confinement of fast ions. The implications of resonant magnetic perturbation on fast ions have been studied in DIII-D, ASDEX Upgrade, and ITER, as referenced in \cite{van2015,sanchis2018,ward2022}. At present, there is no definitive set of criteria for ELM suppression. Therefore, the ELM suppression coils for STEP should be flexible enough to experimentally determine an optimal configuration. Consequently, it is essential to model fast ion losses over a variety of potential resonant magnetic perturbation scenarios. This research will focus on this important modelling effort.

This research will only consider the plasma during the flat-top phase, as this is when the $\alpha$-particle power is at its peak. Nevertheless, it is also essential to investigate the alpha particle confinement during the ramp-up and ramp-down phases in the near future.

To model the the $\alpha$-particles we need detailed information on the temperature, density and background magnetic field. We have predicitions of the flat-top temperature, density and 2D magnetic field which have been provided to us by the STEP integrated modelling team which used by the integrated modelling suite JINTRAC \cite{mitchell2023}.
\begin{itemize}
    \item Talk about how background plasma is calculated.
\end{itemize}

The structure of this paper is as follows: Sec.\ref{sec:locust_work} details our efforts to model alpha-particle confinement in the flat-top scenario and compute the associated losses and heat load on the wall using the LOCUST code. In this section, background quantities such as the magnetic field are held constant. Conversely, in Sec. \ref{sec:halo_work}, we incorporate the magnetic field's response to alpha particles and assess the Toroidal Alfven Eigenmodes (TAE) stability of the plasma using the HALO code. Both Sec. \ref{sec:locust_work} and Sec. \ref{sec:halo_work} start with a presentation and explanation of the model and input files used, followed by the results obtained from our simulations. Finally, in Sec. \ref{sec:discussion_and_conclusions}, we summarize our findings and delve into their implications.

\section{Alpha particle confinement}
\label{sec:locust_work}

The purpose of this section is to calculate the $\alpha$-particle flux that strikes the walls of STEP and determine the peak flux location and magnitude. To gain a better understanding of how our results are affected by different design choices and parameters, we will run simulations with varying parameters and analyze the particle losses. This analysis will be beneficial in the design process, allowing us to reduce the heat load on the reactor walls as much as possible.

\subsection{Model}

We will calculate the $\alpha$-particle energy flux on the walls of STEP using the LOCUST code. We provide a brief description of the code here, but we suggest reading \cite{akers2018, ward2021} for a more comprehensive explanation of how it works.

LOCUST is a Monte Carlo simulation code. It tracks particles, represented by markers, that illustrate samples of the $\alpha$-particle distribution. These markers are traced from their creation until one of two events occur: either they escape and collide with the walls of the tokamak, or their velocity reduces below a certain threshold (namely, 1.5 times the bulk ion temperature). Once their velocity drops below this limit, the particles are considered to have thermalised.

Consider the motion of a marker in a magnetic field. We number each marker with an integer $j \in \mathbb{N}$, and we represent the position of the $j^{\text{th}}$ marker at time $t$ by $\textbf{r}_j(t)$.
We represent the background magnetic field at the position of the marker by $\textbf{B}(\textbf{r}_j)$.
The forces on the marker due to collisions with the background ions and electrons are denoted by $\textbf{F}_{\alpha,i}(\textbf{r}j)$ and $\textbf{F}{\alpha,e}(\textbf{r}_j)$, respectively.
Also, let $\alpha$ denote a charged particle with mass $m_\alpha$ and charge $q_\alpha$.
Then the equation of motion for the $j^{\text{th}}$ marker is given by:
\begin{equation}
m_\alpha\dv[2]{\textbf{r}_j}{t} = q\alpha\dv{\textbf{r}_j}{t}\cross\textbf{B}(\textbf{r}j) + \textbf{F}{\alpha,i}(\textbf{r}j) + \textbf{F}{\alpha,e}(\textbf{r}_j).
\end{equation}

Since we are modelling the the fast ions during the flat-top/steady-state phase, the background magnetic field and 

\subsection{Results}

\lipsum[4-5]

\section{TAE stability calculations}
\label{sec:halo_work}

\subsection{Model}

\subsection{Results}

\lipsum[6-7]

\section{Disussion and conclusions}
\label{sec:discussion_and_conclusions}

Our research confirms that, as anticipated, the confinement of alpha particles is proficient in an axisymmetric system. However, this confinement efficiency deteriorates when the ripple field - arising from the use of a finite number of toroidal field (TF) coils - is introduced. Nonetheless, this poses only minor concerns if we opt to position the TF coils inside the vacuum vessel. Given the current plans to locate the TF coils outside the vacuum vessel, with a major radius greater than approximately 9 meters, this is not expected to present significant problems.

The Edge Localised Mode (ELM) mitigation coils pose a more substantial challenge. Our work demonstrates that if the phase between the upper and lower sets of coils is not optimally chosen, it can result in severe complications. Additional research into ELM suppression, particularly for a device with STEP's dimensions, is necessary. Nevertheless, we believe that the work presented here provides valuable insights to inform the design process. This would enable us to find a solution that effectively mitigates ELMs while also maintaining low fast particle losses.

\section*{Acknowledgements}

This work has been funded by STEP, a UKAEA program to design and build a prototype fusion energy plant and a path to commercial fusion.

% Format text for bibliography
% If more than three authors put et al.
\fontsize{9}{12}\selectfont
\setlength{\parskip}{0pt}
\begin{thebibliography}{9}

\bibitem{nuttall2020} % book chapter
% example: CHAPTER-AUTHOR, A., “Title of chapter in sentence case”, Book Title in Title Case, Publisher, Place of Publication (Year).
    WILSON, H., CHAPMAN, I., DENTON, T., et al., 
    ``STEP---on the pathway to fusion commercialization", 
    Commercialising Fusion Energy, 
    IOP Publishing, 
    (2020).

\bibitem{meyer2023} % poster at conference
% example: PRESENTER, A., “Title of presentation in sentence case”, Paper No., paper presented at Organization seminar on subject, Location, year.
    MEYER, H.,
    ``The plasma scenarios for the Spherical Tokamak for Energy Production (STEP) and their technical implications",
    29\textsuperscript{th} IAEA Fusion Energy Conference,
    poster presentation, 
    London, UK, 
    2023

\bibitem{belova2015} % journal article
% example: AUTHOR, A., AUTHOR, B., AUTHOR, C., Journal article title in sentence case, Abb. J. Title 1 2 (Year) 120–123.
    BELOVA, E., GORELENKOV, N., FREDRICKSON, E., et al., 
    Coupling of neutral-beam-driven compressional Alfv\'en eigenmodes to kinetic Alfv\'en waves in NSTX tokamak and energy channeling, 
    Phys. Rev. Lett. 
    \textbf{115} 1 
    (2015) 
    015001.

\bibitem{freethy2023} % poster at conference
% example: PRESENTER, A., “Title of presentation in sentence case”, Paper No., paper presented at Organization seminar on subject, Location, year.
    FREETHY, S.,
    ``The STEP microwave heating and current drive system",
    29\textsuperscript{th} IAEA Fusion Energy Conference,
    poster presentation, 
    London, UK, 
    2023

\bibitem{mitchell2023} % journal article
% example: AUTHOR, A., AUTHOR, B., AUTHOR, C., Journal article title in sentence case, Abb. J. Title 1 2 (Year) 120–123.
    MITCHELL, J., PARROTT, A., CASSON, F., et al.,
    Scenario trajectory optimization and control on STEP,
    Fusion Eng. Des.
    \textbf{192} 
    (2023) 
    113777.

\bibitem{zsolt2023} % internal report
% example: AUTHOR, A., Internal Report Title in Title Case, internal report, Organization, Location, Year.
    ZSOLT, V., et al., 
    TD-001004, 
    internal report, 
    UKAEA, 
    2023.

\bibitem{fil2023} % poster at conference
% example: PRESENTER, A., “Title of presentation in sentence case”, Paper No., paper presented at Organization seminar on subject, Location, year.
    FIL, A., HENDEN, L., NEWTON, S., et al.,
    ``Disruption runaway electron generation and mitigation in the Spherical Tokamak for Energy Production",
    poster presentation, 
    29\textsuperscript{th} IAEA Fusion Energy Conference,
    London, UK, 
    2023

\bibitem{zohm1996} % journal article
% example: AUTHOR, A., AUTHOR, B., AUTHOR, C., Journal article title in sentence case, Abb. J. Title 1 2 (Year) 120–123.
    ZOHM, H., 
    Edge localized modes (ELMs), 
    Plasma Phys. Control. Fusion 
    \textbf{38} 2 
    (1996) 
    105.

\bibitem{wagner1982} % journal article
% example: AUTHOR, A., AUTHOR, B., AUTHOR, C., Journal article title in sentence case, Abb. J. Title 1 2 (Year) 120–123.
    WAGNER, F., BECKER, G., BEHRINGER, K., et al., 
    Regime of improved confinement and high beta in neutral-beam-heated divertor discharges of the ASDEX tokamak, 
    Phys. Rev. Lett. 
    \textbf{49} 19
    (1982) 
    1408.

\bibitem{suttrop2018} % journal article
% example: AUTHOR, A., AUTHOR, B., AUTHOR, C., Journal article title in sentence case, Abb. J. Title 1 2 (Year) 120–123.
    SUTTROP, W., KIRK, A., BOBKOV, V., et al.,
    Experimental conditions to suppress edge localised modes by magnetic perturbations in the ASDEX Upgrade tokamak,
    Nucl. Fusion,
    \textbf{58} 9 
    (2018) 
    096031.
    
\bibitem{van2015} % journal article
% example: AUTHOR, A., AUTHOR, B., AUTHOR, C., Journal article title in sentence case, Abb. J. Title 1 2 (Year) 120–123.
    VAN ZEELAND, M., FERRARO, N., GRIERSON, B., et al.,
    Fast ion transport during applied 3D magnetic perturbations on DIII-D,
    Nucl. Fusion,
    \textbf{55} 7,
    (2015)
    073028.

\bibitem{sanchis2018} % journal article
% example: AUTHOR, A., AUTHOR, B., AUTHOR, C., Journal article title in sentence case, Abb. J. Title 1 2 (Year) 120–123.
    SANCHIS, L., GARCIA-MUNOZ, M., SNICKER, A., et al.,
    Characterisation of the fast-ion edge resonant transport layer induced by 3D perturbative fields in the ASDEX Upgrade tokamak through full orbit simulations,
    Plasma Phys. Control. Fusion,
    \textbf{61} 1,
    (2018)
    014038.
    
\bibitem{ward2022} % journal article
% example: AUTHOR, A., AUTHOR, B., AUTHOR, C., Journal article title in sentence case, Abb. J. Title 1 2 (Year) 120–123.
    WARD, S., AKERS, R., LI, L., et al.,
    LOCUST-GPU predictions of fast-ion transport and power loads due to ELM-control coils in ITER,
    Nucl. Fusion,
    \textbf{62} 12
    (2022)
    126014.

\bibitem{akers2018} % poster at conference
    % example: PRESENTER, A., “Title of presentation in sentence case”, Paper No., paper presented at Organization seminar on subject, Location, year.
    AKERS, R., COLLING, B., HESS, J., et al.,
    ``High fidelity simulations of fast ion power flux driven by 3D field perturbations on ITER",
    poster presentation, 
    26\textsuperscript{th} IAEA Fusion Energy Conference,
    Kyoto, Japan, 
    2016.

\bibitem{ward2021} % journal article
% example: AUTHOR, A., AUTHOR, B., AUTHOR, C., Journal article title in sentence case, Abb. J. Title 1 2 (Year) 120–123.
    WARD, S., AKERS, R., JACOBSEN, A., et al.,
    Verification and validation of the high-performance Lorentz-orbit code for use in stellarators and tokamaks (LOCUST),
    Nucl. Fusion,
    \textbf{61} 8
    (2021)
    086029.






\end{thebibliography}


\end{document}
