
\documentclass[10pt, a4paper, twoside]{article}
\usepackage[utf8]{inputenc}
\usepackage{newtxtext,newtxmath} % Times New Roman font
\usepackage[left=2.54cm, right=2.54cm, top=2.54cm, bottom=2.54cm, headheight=15pt]{geometry}
\usepackage{titlesec} % For customizing section titles
\usepackage{enumitem} % Enables bullet points
\usepackage{lipsum} % Add dummy text
\usepackage{hyperref} % Links for references
\usepackage{parskip} % Set spacing between paragraphs
\usepackage{fancyhdr} % Format headers
\usepackage{etoolbox} % Format bibliography header
\usepackage{siunitx} % Used to write SI base units
\usepackage{physics} % Used to write differential equations more easily
\usepackage{graphicx} % Used for figures
\usepackage{caption} % Used to edit the figure captions
\usepackage{ulem} % Enables strikethrough text with \sout command.

% This code is used to supress \qty warning conflict between siunitx and physics packages
\ExplSyntaxOn
\msg_redirect_name:nnn { siunitx } { physics-pkg } { none }
\ExplSyntaxOff

\setlength{\parskip}{10pt}

\titleformat{\section}{\normalfont\fontsize{10}{13}\uppercase}{\thesection.}{1em}{}
\titleformat{\subsection}{\normalfont\fontsize{10}{13}\bfseries}{\thesubsection.}{1em}{}
\titleformat{\subsubsection}{\normalfont\fontsize{10}{13}\itshape}{\thesubsubsection.}{1em}{}

% Redefining \section* command for centered section title
\titleformat{name=\section,numberless}{\normalfont\fontsize{10}{13}\bfseries\centering\uppercase}{}{0pt}{}

% Ensure references title has the same format as \section*
\patchcmd{\thebibliography}{\section*{\refname}}{\section*{REFERENCES}}{}{}

%  Create headers and footers
\fancyhf{} % clear all header and footer fields
\renewcommand{\headrulewidth}{0pt}
% \fancyhead[CE]{\fontsize{8}{10}\selectfont \textbf{IAEA-CN-316/2143}}
\fancyhead[CE]{\fontsize{8}{10}\selectfont \textbf{CONFINEMENT OF ALPHAS IN STEP}}
\fancyhead[CO]{\fontsize{8}{10}\selectfont \textbf{PROKOPYSZYN et al.}}
\fancyfoot[RO]{\fontsize{10}{13}\selectfont \thepage}
\fancyfoot[RE]{\fontsize{10}{13}\selectfont \thepage}
\pagestyle{fancy}

% Set the caption font size to 9pt
\DeclareCaptionFont{ninept}{\fontsize{9pt}{12pt}\selectfont}
\renewcommand{\figurename}{\textit{FIG.}}
\captionsetup{
    justification=raggedright,
    singlelinecheck=true,
    font=ninept,
    textfont=it,
    labelfont=it,
    labelsep=period,
    labelformat=simple,
    format=hang
}

\begin{document}
\begin{flushleft}
\fontsize{12}{14}\selectfont \textbf{CONFINEMENT OF FUSION ALPHA-PARTICLES AND ALFV\'EN EIGENMODE STABILITY IN STEP}
% \fontsize{12}{14}\selectfont \textbf{\textit{Subtitle if needed in Times New Roman 12 point bold italic, sentence case}}

\fontsize{10}{13}\selectfont
A.P.K. PROKOPYSZYN, K.G. MCCLEMENTS, H.J.C. OLIVER, M. FITZGERALD, D.A. RYAN, G. XIA \\
United Kingdom Atomic Energy Authority \\
Culham Campus, Abingdon, Oxfordshire OX14 3DB, United Kingdom \\
Email: alex.prokopysyzn@ukaea.uk

\end{flushleft}

% Abstract section
\begin{flushleft}
\textbf{Abstract}
\end{flushleft}

\setlength{\parindent}{1cm}
\fontsize{9}{12}\selectfont

The Spherical Tokamak for Energy Production (STEP) programme is focused on designing and building a prototype fusion power plant that will generate approximately 1.5-1.8 GW of deuterium-tritium fusion power. To achieve this, the $\alpha$-particles generated through fusion must be adequately confined to maintain the necessary high temperature in the core of the plasma and to protect the wall from excessive damage. Microwaves will be used for both external heating and current drive, making $\alpha$-particles the only significant fast-ion species. The purpose of this work is to model the confinement of $\alpha$-particles and the toroidal Alfvén eigenmodes (TAEs) driven by these particles in a variety of scenarios to help determine the best configuration. The scenarios examined here have been identified by the STEP team as potential flat-top operating configurations. We use LOCUST (Lorentz Orbit Code for Use in Stellerators and Tokamaks) to model the $\alpha$-particle confinement and heat-load distribution on the wall, and HALO (HAgis LOcust) to model the TAEs. The results indicate that acceptable confinement in terms of power loading can be achieved in candidate flat-top operating points, but the results are sensitive to some of the system parameters. For example, a change in the phase difference between the upper and lower edge localised mode (ELM) suppression coils can increase the maximum power load on the first wall due to $\alpha$-particle losses by a factor of 10.

% Abstract and title over, format text for the main body
\setlength{\parindent}{0pt}
\fontsize{10}{13}\selectfont

\section{Introduction}
\label{sec:introduction}

UKAEA is pioneering efforts to develop a compact prototype of a fusion energy power plant and establish a commercial pathway for fusion energy \cite{nuttall2020, meyer2023, mitchell2023}. A crucial factor in the viability of a deuterium-tritium (D-T) burning fusion reactor is the confinement of fusion-born $\alpha$-particles since, in addition to providing most of the plasma heating, these particles have the potential to damage the reactor walls when unconfined. Ideally, $\alpha$-particles should slow down through Coulomb collisions with thermal electrons and ions while undergoing a modest level of (purely collisional) transport. However, they can drive high-frequency instabilities that, in turn, can transport them at rates well above collisional levels \cite{garcia-munoz2011}. Toroidal Alfv\'en eigenmodes (TAEs) exemplify the type of fast particle-driven instability that can significantly disrupt plasma confinement. In the present work we model $\alpha$-particle losses in the presence of static magnetic fields and determine the stability of TAEs in candidate STEP plasmas with the objective of informing the design process for this device.

External heating and current drive for the plasma in STEP during the flat-top configuration will be provided entirely by microwaves. This heating strategy will utilise a combination of electron cyclotron current drive and electron Bernstein waves. Detailed information on the proposed heating and current drive systems for STEP is provided in \cite{Henderson2024}.
This approach has significant implications, as it means that the only substantial population of fast ions will be composed of $\alpha$-particles, which are a product of the fusion reaction itself. Unlike other tokamak devices such as JET and ITER, STEP will not generate fast ions via neutral beam injectors or ion cyclotron resonance heating systems.
In this context, `fast ions' are defined as ions with speeds that are significantly greater than those of the background plasma, making them suprathermal. For example, in the STEP design, the plasma at the core is expected to reach a temperature of approximately 20 keV (see, e.g. \cite{meyer2023,mitchell2023}), whereas the fusion-born $\alpha$-particles will be born with energies of around 3.5 MeV, much higher than the background plasma.

The commercial viability of STEP will rely on the longevity of the plasma-facing components. During operation, the PFCs (plasma facing components) will be exposed to heat from various sources, even if disruptions are successfully avoided. The wall will be exposed to electromagnetic radiation from the plasma, neutrons, thermal charged particles, runaway electrons, erosion due to sputtering, and $\alpha$-particles.
In \cite{vaccaro2024}, the total exhausted power to the plasma-facing components (PFCs) for STEP is estimated to be 485 MW. They consider all heat sources in their work and estimate that peak fluxes can be as high as approximately 1\,$\si{MW.m^{-2}}$. For reference, in this work, we estimate that the heat load on the PFCs due to fast thermonuclear alpha particles can reach up to approximately 40 MW. It is important to note that the alpha heat load can be highly localised. In this study, our highest estimate for peak alpha particle heat loads is about 1\,$\si{MW.m^{-2}}$. However, more realistic simulations give peak fluxes around 0.1\,$\si{MW.m^{-2}}$.
See, for example, \cite{bachmann2018} for a discussion {\it inter alia} of the heat loads on PFCs in a conventional tokamak reactor and \cite{Berger2022} for a study of runaway electron dynamics in STEP-like plasmas undergoing disruptions. Our objective is to maintain the heat load on the wall at a level less than approximately $1\, \text{MWm}^{-2}$ on the first wall of the main chamber, about $5\, \text{MWm}^{-2}$ on the dome structures in the divertor regions, and $10\, \text{MWm}^{-2}$ on the divertors themselves. Given that the design of the PFCs in STEP is still in its early stages, the plasma-facing components in our model are primarily smooth, resulting in the neglect of tile shadowing effects. Consequently, we should aim for somewhat lower energy fluxes to compensate for shadowing, which can increase the maximum heat load.

Because of their large energies, the confinement of $\alpha$-particles can easily be compromised due to deviations in the background magnetic field from axisymmetry.  In this study, we model $\alpha$-particles in fields that are axisymmetric, except for three types of three-dimensional perturbation. The first of these stems from the ripple introduced by the use of a finite number of toroidal field (TF) coils. The second arises from coils that generate resonant magnetic perturbations (RMPs) aimed at suppressing edge-localized modes (ELMs) \cite{zohm1996}. The last three-dimensional field comes from the RWM active feedback control coils used to suppress resistive wall modes (RWMs) \cite{xia2023}.

STEP will need to operate in high-confinement mode (H-mode), and therefore type-I ELMs may pose a significant problem. ELM suppression using RMP coils has been successfully demonstrated experimentally in several medium-sized conventional tokamaks, including DIII-D \cite{Evans2008}, ASDEX Upgrade \cite{suttrop2018} and KSTAR \cite{In2019}. However, the nonaxisymmetric RMP fields violate conservation of fast-ion toroidal canonical momentum and can thus deconfine these ions. The effects of RMPs on fast ions have been studied in DIII-D, ASDEX Upgrade, and ITER \cite{van2015,sanchis2018,ward2022}. Currently, there is no definitive set of criteria for ELM suppression. Therefore, the ELM suppression coils for STEP should be flexible enough to experimentally determine an optimal configuration, and it is necessary to model $\alpha$-particle losses for a range of RMP scenarios.

We only consider the plasma during the flat-top phase as $\alpha$-particle production will be negligible for the greater part of the ramp-up and ramp-down phases. In order to accurately model the $\alpha$-particles we require profiles of temperature and density as well as the magnetic field. These profiles have been determined using the integrated modelling suite JINTRAC \cite{meyer2023, mitchell2023}. This sophisticated tool incorporates a wide range of physics processes, including simplified fast-ion models, providing a comprehensive basis for our analysis.

The structure of this paper is as follows. Section \ref{sec:locust_work} details our efforts using the LOCUST code to model $\alpha$-particles in the flat-top and compute the associated heat load on the wall. In this section, background quantities, including the magnetic field, are assumed to be constant in time. In Section \ref{sec:halo_work} we consider the magnetic field's response to the presence of $\alpha$-particles and assess TAE stability using the HALO code. In Section \ref{sec:discussion_and_conclusions} we summarise our findings and discuss their implications.

\section{Alpha particle confinement}
\label{sec:locust_work}

Our goal in this section is to calculate the $\alpha$-particle flux that strikes the walls of STEP and to determine the location and magnitude of the peak flux. To gain a better understanding of how our results are affected by different design choices and parameters, we have run simulations with varying parameters and analysed the particle losses. This analysis is beneficial in the design process, allowing us to identify scenarios that minimise the heat load on the reactor walls.

\subsection{Model}
\label{sec:model}

We calculate the $\alpha$-particle energy flux on the walls of STEP using the LOCUST code (see \cite{akers2018, ward2021} for a more comprehensive explanation of how it works). This tracks particles, represented by markers that sample the $\alpha$-particle distribution. These markers are traced from birth until they escape and collide with PFCs, or their energy drops below a threshold equal to 1.5 times the bulk ion temperature, at which point they are considered to have thermalised. 

We model the backgound quantities as fixed. Fig. \ref{fig:background_plasma_profiles} shows some profiles of the background plasma as a function of the normalised poloidal flux $\Psi_N$. 
The top left of Fig. \ref{fig:background_plasma_profiles} shows temperature profiles for the ions ($T_i$) and electrons ($T_e$) in keV. 
The top right shows the safety factor profile. 
The bottom left shows the number density profiles of all the background particle species: electrons ($n_e$), deuterium ($n_D$), tritium ($n_T$), xenon ($n_{Xe}$) and helium ($n_{He}$) in $\si{m^{-3}}$. 
The bottom right shows the Bosch-Hale D-T reaction rate \cite{bosch1992} in $\si{m^{-3}.s^{-1}}$. 
Similarly, some of the key values of the background plasma are given in Table \ref{table:fusion_params}. Note that  $R_{\text{axis}}$ represents the major radius of the magnetic axis, $B_{\text{axis}}$ denotes the magnetic field strength at the magnetic axis, $I_{\text{p}}$ is the total plasma current, $Z_{\text{eff}}$ is the effective atomic number of the plasma, $P_{\text{fusion}}$ is the total fusion power, and $P_{\alpha}$ represents the portion of the fusion power carried by thermonuclear alpha particles.
It is important to emphasise that the STEP team is still in the process of designing STEP, and the final design will likely use different values than those presented here. In Appendix \ref{appendix:sensitivity_of_results_to_background_equillibria}, we analyze results in a slightly different equilibrium with more impurities and demonstrate that this leads to a threefold increase in the peak heat load. Therefore, it is crucial to note that these results pertain to just one background equilibrium, so caution must be exercised when extrapolating to other scenarios.

\begin{figure}[!ht]
    \centering
    \includegraphics[width=0.99\linewidth]{Figures/background_plasma_curves.pdf}
    \caption{Profiles of key background plasma quantities as a function of the normalised poloidal flux, $\psi_N$.}
    \label{fig:background_plasma_profiles}
\end{figure}

\begin{table}[htbp]
\centering
\begin{tabular}{ccccccc}
\hline
$R_{\text{axis}}$ & $B_{\text{axis}}$ & $I_{\text{p}}$ & $Z_{\text{eff}}$ & $P_{\text{fusion}}$ & $P_{\alpha}$ \\
\hline
4.38 m & 2.45 T & 20.10 MA & 2.73 & 1.66 GW & 0.33 GW \\
\hline
\end{tabular}
\caption{Key quantities of the background plasma model.}
\label{table:fusion_params}
\end{table}

The initial position of the markers we use to represent samples of the thermonuclear alpha particle distribution function are sampled randomly and uniformly across the plasma volume. However, we assign a weight to each marker equal to the Bosch-Hale D-T reaction rate (see bottom right of Fig. \ref{fig:background_plasma_profiles}). We model the birth velocity distribution as isotropic. The kinetic energy, $E_j$, of the $j^{\text{th}}$ marker is sampled from the following normal distribution (see \cite{brysk1973})
\begin{equation}
    E_j \sim N\qty(\text{E}(E_j), 2\frac{m_\alpha}{m_\alpha + m_n} T_i\text{E}(E_j) ),
\end{equation}
where $\text{E}(E_j)=3.5\, \si{MeV}$ is the mean birth energy of the $\alpha$-particles from the nuclear fusion, $m_\alpha$, is the mass of an $\alpha$-particle, $m_n$ is the mass of a neutron, $T_i$ is the background temperature of the ions in $\si{eV}$. In the core, the standard deviation of the energy is about 10\%.

We represent the position of the $j^{\text{th}}$ marker at time $t$ by $\mathbf{x}_j(t)$, and denote
the background magnetic field at the position of the marker by $\textbf{B}(\mathbf{x}_j)$.
The forces on the marker due to collisions with the background ions and electrons are $\textbf{F}_{\alpha,i}(\mathbf{x}_j)$ and $\textbf{F}_{\alpha,e}(\mathbf{x}_j)$ respectively, while the mass and charge of the $\alpha$-particles are $m_\alpha$ and $q_\alpha$ respectively. The equation of motion for the $j^{\text{th}}$ marker is
\begin{equation}
m_\alpha\dv[2]{\mathbf{x}_j}{t} = q_\alpha\dv{\mathbf{x}_j}{t}\cross\textbf{B}(\mathbf{x}_j) + \textbf{F}_{\alpha,i}(\mathbf{x}_j) + \textbf{F}_{\alpha,e}(\mathbf{x}_j).
\end{equation}

We do not consider collisions between fast $\alpha$-particles, an omission that is justified by the low concentration of this species, and consequently each $\alpha$-particle is modelled independently. This approach presents substantial computational advantages as it allows for parallel processing. To capitalise on this, LOCUST utilises Graphics Processing Unit (GPU) architecture, enabling the simultaneous modelling of very large numbers of particles. This approach is computationally more efficient and cost-effective than Central Processing Unit (CPU)-based modelling.

Friction forces in the system are computed using a Monte Carlo Fokker-Planck collision operator that includes terms for diffusion and drift of pitch angle and energy. The collision frequencies are derived under the assumption of a weakly-coupled plasma and Maxwellian velocity distribution. For more information, see \cite{ward2021}.

The axisymmetric component of the magnetic field was calculated using the free-boundary FIESTA code. Fig. \ref{fig:coil_plot_3d} shows the last closed-flux surface of the axisymmetric field. Additionally, three types of 3D field with a magnitude much smaller than that of the 2D field are included. Section \ref{sec:tf_ripple_field} models the ripple field produced by the TF coils. Section \ref{sec:elm_suppression_field} calculates the impact of the ELM mitigation field on the confinement of the $\alpha$-particles. Section \ref{sec:rwm_field} calculates the impact of a resistive wall mode on the confinement of the $\alpha$-particles. Finally, Section \ref{sec:all_3d_fields} analyses the energy distribution in a simulation where all the aforementioned 3D magnetic field contributions are superimposed.

\begin{figure}[htpb]
    \centering
    \includegraphics[width=0.8\linewidth]{Figures/coil_plot_3d.pdf}
    \caption{Schematic representation of STEP’s last closed flux surface, ELM suppression coils (which are interior and exterior to the vacuum vessel), RWM active control coils and TF coils. These designs are likely to change in the future. 
}
    \label{fig:coil_plot_3d}
\end{figure}

At the end of a LOCUST simulation we calculate the power lost to the first wall using the expression 
\begin{equation} 
P_{\text{lost}} = P_\alpha \frac{\sum_{j\in \{\text{set of indices for the markers that hit the wall}\}}\abs{\textbf{v}_{\text{final}, j}}^2w_j}{\sum_{j \in \{\text{set of all marker indices}\}} \abs{\mathbf{v}_{\text{initial}, j}}^2w_j}, 
\end{equation}
where $P_\alpha$ is the thermonuclear $\alpha$-particle power, equal to $\si{330.MW}$ (which arises from a fusion power of $1.\si{66.GW}$) in the present case. Additionally, $\mathbf{v}_{\text{final}, j}$,  is the final velocity of the j\textsuperscript{th} marker, $\mathbf{v}_{\text{initial}, j}$ is the birth velocity of the j\textsuperscript{th} marker and $w_j$ is the weight of the j\textsuperscript{th} marker. Note that the weight, $w_j$, is given by the Bosch-Hale DT reaction rate (see bottom right of Fig. \ref{fig:background_plasma_profiles}).

To assess whether the Plasma-Facing Components (PFCs) can withstand the impacts of thermonuclear $\alpha$-particles, it is crucial to determine the energy flux of these particles on the wall, which is not uniformly distributed. The LOCUST simulations provide us with the positions and energies at which the markers hit the PFCs. However, since the energy flux is a continuous function and the markers offer only discrete data, we employ a technique described in \cite{chen2017} and further elaborated in Appendix \ref{appendix:kernel_density_estimation} to calculate the energy flux (measured in $\si{W.m^{-2}}$) across the wall. This technique utilises kernel density estimation (KDE), with bandwidths determined through cross-validation and bootstrap resampling to compute 95\% confidence intervals.

In Section \ref{sec:elm_suppression_field}, we model the plasma's response to the magnetic field generated by the ELM suppression coils. In Section \ref{sec:rwm_field}, we include the magnetic field generated by the growth of a resistive wall mode (RWM). Both of these scenarios are analyzed using the MARS-F code \cite{liu2015}, which solves the linearized MHD equations. MARS-F can operate using either an eigenvalue approach or as an initial value problem. For modeling the plasma response to the ELM suppression coils, we employed the initial value problem approach. In contrast, for the RWM, we used the eigenvalue approach.

% provides us with discrete data of where We use a technique called kernel density estimation to calcualte a, we can parameterise it using the coordinates $s_\theta$ and $\phi$, where $s_\theta$ is the poloidal distance along the wall in metres and $\phi$ is the toroidal coordinate in radians. At the end of a LOCUST simulation we have the locations, energies and weights of markers that hit the wall in the $\phi$-$s_\theta$ plane. We use kernel density estimation with Gaussian kernels to prudce a smooth function in the $\phi$-$s_\theta$ plane. Taking suitable derivatives and multiplying by the appropriate Jacobian we can calculate the 
% We calculate the energy flux using kernel density estimation \cite{chen2017} and leave-one-out cross-validation \cite{chen2017} to determine the bandwidth. We also use bootstrap resampling to calculate 95\% confidence intervals for the maximum flux \cite{chen2017}. 

\subsection{Results}

\subsubsection{TF ripple field}
\label{sec:tf_ripple_field}

It is expected that STEP plasmas will be highly elongated. The toroidal field (TF) coils of the current design are projected to be around $\si{22.m}$ tall and have a rectangular shape (see Fig. \ref{fig:coil_plot_3d}), although this may be altered in the future. Due to the coil geometry, the ripple field in the plasma can be well-approximated by
% \begin{equation}
%     \label{eq:ripple_field}
%     \begin{aligned}
%         \delta B_R^{\text{ripple}}(R, \phi, z) &\approx \frac{B_0 R_0}{R} \qty(\frac{R}{R_{\text{coil}}})^{N_{coil}}\sin(N_{\text{coil}}\phi), \\
%         \delta B_\phi^{\text{ripple}}(R, \phi, z) &\approx \frac{B_0 R_0}{R} \qty(\frac{R}{R_{\text{coil}}})^{N_{coil}}\cos(N_{\text{coil}}\phi),\\
%         \delta B_z^{\text{ripple}}(R, \phi, z) &\approx 0.
%     \end{aligned}
% \end{equation}
\begin{equation}
    \label{eq:ripple_field}
    \begin{aligned}
        \delta B_R^{\text{ripple}}(R, \phi, z) &\approx \frac{B_\text{axis} R_\text{axis}}{R}\qty[\qty(\frac{R}{R_{\text{outer}}})^{N_\text{coil}}-\qty(\frac{R_\text{inner}}{R})^{N_\text{coil}}]\sin(N_{\text{coil}}\phi), \\
        \delta B_\phi^{\text{ripple}}(R, \phi, z) &\approx \frac{B_\text{axis} R_\text{axis}}{R}\qty[\qty(\frac{R_\text{inner}}{R})^{N_\text{coil}} + \qty(\frac{R}{R_{\text{outer}}})^{N_\text{coil}}]\cos(N_{\text{coil}}\phi),\\
        \delta B_z^{\text{ripple}}(R, \phi, z) &\approx 0.
    \end{aligned}
\end{equation}
Here, $R$ denotes the major radius, $R_\text{axis}$ is the major radius of the magnetic axis of the plasma, $B_\text{axis}$ represents the background toroidal magnetic field at $R=R_0$, $N_{\text{coil}}$ is the number of TF coils, $R_\text{inner}$, $R_\text{outer}$ are the major radii of the inner and outer legs of the TF coils respectively and $\phi$ is toroidal angle. A derivation of Equation \eqref{eq:ripple_field} can be found in, for example, \cite{mcclements2005}. In Appendix \ref{appendix:tf_ripple_analytic_approximation} we assess the accuracy of the above formula by comparing it with numerically calculated fields and show that the error is within 1\%.
% When we compared Equation \eqref{eq:ripple_field} with the numerical predictions of the ripple field, the discrepancy was below 1\% on the outer side of the plasma. Furthermore, our simulations, using both methodologies, confirmed concordant results within the statistical error margins arising from the use of a finite number of markers.

Fig. \ref{fig:max_and_total_flux_vs_rcoil_and_ncoil} shows the maximum energy flux on the walls and the total flux for different radii of the TF coils' outer limb, $R_\text{outer}$, and the number of TF coils, $N_\text{coil}$. 
Note that for these simulations, we used an inner radius of $R_\text{inner}=0.75\,\si{m}$. In Appendix \ref{appendix:tf_ripple_analytic_approximation}, we showed that the results are insensitive to the choice of $R_\text{inner}$ for $R_\text{inner}<1\,\si{m}$. A value of $R_\text{inner}=0.75\,\si{m}$ is towards the higher end of what is realistic; the current design of STEP is closer to about $0.5\,\si{m}$. 
The current design, with a major radius of approximately $\si{9.m}$ and 16 TF coils, has power losses that remain within acceptable limits. Increasing the radius of the coil or the number beyond these values will have a minimal effect on improving the confinement of the $\alpha$ particles. 
% Unfortunately, due to the need to fit other instruments such as the vacuum vessel, it is difficult to reduce the outer radius of the TF coil. The results also show that we can reduce the number of TF coils from 16 to 12 without affecting the confinment (assuming a radius of $9\si{.m}$ is maintained).
Unfortunately, it is difficult to save costs by reducing the outer radius of the TF coils due to the need to accommodate other essential elements such as the vacuum vessel, blanket, and PF (polodial field) coils. However, the results indicate that the number of TF coils could be reduced from 16 to 12 without compromising the confinement if a coil radius of $\si{9.m}$ were adopted.

\begin{figure}[htpb]
    \centering
    \vspace{1cm}
    \includegraphics[width=0.99\linewidth]{Figures/max_and_total_flux_vs_rcoil.pdf}
    \caption{TF ripple-induced losses for three values of $N_{coil}$ (the number of TF coils). The left plot shows the maximum energy flux on the reactor wall in $\si{MW.m^{-2}}$ and the right plot indicates the percentage of $\alpha$-particle power escaping and impacting the PFCs. Error bars represent 95\% confidence intervals, reflecting the statistical uncertainty inherent in the Monte Carlo methodology of the simulation. The black horizontal dotted line shows the results from a simulation where we used only the axisymmetric field.}
    \label{fig:max_and_total_flux_vs_rcoil_and_ncoil}
\end{figure}

\newpage
\subsubsection{Internal and External ELM suppression fields}
\label{sec:elm_suppression_field}

In this section we model the confinement of $\alpha$-particles in the presence of ELM suppression fields.
% , excluding the TF ripple component (the latter has a minimal effect on the confinement in the most recent TF coil design). 
The ELM suppression field can be generated by coils that are inside the vacuum vessel (see blue coils in Fig. \ref{fig:coil_plot_3d}) or external to the vacuum vessel (see red coils in Fig. \ref{fig:coil_plot_3d}). Ex-vessel coils are currently favoured since it may be challenging to provide sufficient cooling to the in-vessel coils. Nevertheless, a definitive conclusion on which coil set will be used has not been reached, so we will analyse both scenarios in this study.

There are sixteen of each type of ELM suppression coil in each row, as illustrated in Figure \ref{fig:coil_plot_3d}. The current in the $k^\text{th}$ coil of the upper and lower rows is given by
\begin{align}
    \label{eq:ELM_coilcurrent_profile_upper}
    I_k^{upper} &= I_0 \cos(n \phi_k + \Delta \phi), \\
    \label{eq:ELM_coilcurrent_profile_lower}
    I_k^{lower} &= I_0 \cos(n \phi_k)
\end{align}
where $k$ ranges from 1 to 16, $I_0$ is the maximum current value, $\phi_k$ is the toroidal angle of the centre of the $k^\text{th}$ coil, $n$ is the principal toroidal mode number chosen to be excited, and $\Delta \phi$ is a free parameter that provides a phase shift. This produces a magnetic field that can be expressed as
\begin{equation}
    \delta\textbf{B}^\text{ELM}(R, \phi, z)=\Re\qty[\qty(\delta\textbf{B}^\text{ELM}_\text{real}(R,z) + i\delta\textbf{B}^\text{ELM}_\text{imag}(R,z))\exp(in\phi)].
\end{equation}
Other toroidal harmonics (sidebands) are present, but they have a sufficiently small amplitude that their effect on $\alpha$-particle confinement can be ignored when 16 ELM suppression coils are used. However, if only 8 coils are employed, the sidebands cannot be ignored. The magnetic fields arising from these currents are, in general, modified as a result of their interaction with the plasma. Plasma reaction is more prominent for lower toroidal mode numbers ($n$) \cite{mcclements2005}. For the ripple field, the toroidal is mode large enough that we can neglect the plasma response. For the suppression of ELMs, we plan to set $n=3$. Higher values of $n$ decay more quickly with distance from the coils, decreasing their efficacy, while lower values of $n$ may activate locked modes. Nevertheless, the choice of $n$ may change, and so we also model the cases with $n=2$ and $n=4$. To model the plasma response, we used the MARS-F code \cite{liu2015}. It is difficult to confirm the accuracy of the plasma response calculations and so we will also analyse the results for the case where the vacuum ELM suppression field is used.

% Equation \ref{eq:ELM_coilcurrent_profile} has three adjustable parameters: the toroidal mode number ($n$), the amplitude ($I_0$), and the phase shift $\Delta \phi$.
We have calculated the optimal values of $\Delta \phi = \Delta\phi_\text{opt}$, which maximise a quantity denoted by $\abs{b^1_{res}}$. This is a dimensionless measure of the field perturbation perpendicular to a resonant flux surface. 
Note that $b^1$ is defined by the component of the perturbation perpendicular to the background flux surface, divided by the magnitude of the magnetic field at that location.
We choose a flux surface very close to the separatrix, at $q=10$, as the one at which $\abs{b^1_{res}}$ is maximised. 
% as this is the outermost surface that is resonant for all values of $n=2$, 3, 4. 
Note that $\abs{b^1_{res}}$ gives a measure of the displacement of the X-point \cite{ryan2017}. In \cite{suttrop2018}, the authors showed that $\abs{b^1_{res}}=10^{-4}$ was sufficient to suppress ELMs in ASDEX Upgrade. The authors in \cite{liu2015} hypothesise that a similar value of $\abs{b^1_{res}}=10^{-4}$ will be needed to suppress ELMs for ITER. Extrapolating from the aforementioned ASDEX Upgrade results, we hypothesise that a value $\abs{b^1_{res}}=10^{-4}$ will be needed to suppress ELMs for STEP. However, since this is an extrapolation, we will also model the fast ion losses in scenarios where $\abs{b^1_{res}}$ is greater than $10^{-4}$. We present the optimal phase shifts ($\Delta\phi_\text{opt}$) and the corresponding coil currents required to ensure $\abs{b^1_{res}}>10^{-4}$ in Table \ref{table:optimum_currents_and_phases}. 
\begin{table}[htbp]
\centering
\begin{tabular}{lccc}
\hline
 & \( n=2 \) & \( n=3 \) & \( n=4 \) \\
\hline
In-Vessel ELM suppression coil current [kAt] & 30 & 50 & 80 \\
Ex-Vessel ELM suppression coil current [kAt] & 50 & 90 & 150 \\
In-Vessel ELM suppression coil \(\Delta\phi_\text{opt}\) [degrees] & 265 & 173 & 67 \\
Ex-Vessel ELM suppression coil \(\Delta\phi_\text{opt}\) [degrees] & 61 & 20 & 321 \\
\hline
\end{tabular}
\caption{Current in kAt required to exceed \( b^1_{res} > 10^{-4} \), and optimal coil phases.}
\label{table:optimum_currents_and_phases}
\end{table}

To suppress ELMs, $I_0$ must be large enough, but not so large that it reduces $\alpha$-particle confinement to an unacceptable extent. Given the uncertainty on the current required to suppress ELMs we will model the $\alpha$-particle losses with the currents quoted in Table \ref{table:optimum_currents_and_phases} and twice these values.
% Experiments on ASDEX Upgrade, when extrapolated to STEP, suggest that for $n=2$, a current of 50 kAt would be necessary, for $n=3$ a current of 90 kAt would be required, and for $n=4$ a current of 150 kAt would be needed. However, there is a high degree of uncertainty over the current needed, so we also model coil currents with twice these values. 
We model the $\alpha$-particle losses with the phase shifts $\Delta\phi_\text{opt}$ shown in Table \ref{table:optimum_currents_and_phases}. Additionally, we analyse the impact of changing the phase shift $\Delta\phi$ of the current profile of the upper coil set in comparison to the lower set on the confinement of fast particles. We consider $\Delta\phi$ values of $0^\circ$, $45^\circ$, $90^\circ$, $135^\circ$, $180^\circ$, $225^\circ$, $270^\circ$, and $315^\circ$. It should be noted that the $\Delta\phi$ used here is equal to $360^{\circ}-\Delta\phi^{\prime}$ where $\Delta\phi^{\prime}$ is the phase shift referred to in papers that report MARS-F modelling of RMPs, for example \cite{ryan2017}   

Fig. \ref{fig:max_and_total_flux_vs_phase_elm_rwm} and \ref{fig:max_and_total_flux_vs_phase_elm} show the predicted maximum flux of $\alpha$-particle energy on the first wall of STEP and the percentage of $\alpha$-power lost. As in the TF ripple simulations, we assume 1.66 GW of fusion power (hence $\sim\si{330.MW}$ of $\alpha$-particle power). The results are highly sensitive to the phase shift, as observed experimentally in ASDEX Upgrade \cite{sanchis2018}. These two figures also show that the losses are generally greater when the plasma response to the ELM suppression field is taken into account. However even when larger coil currents are used and the plasma response is included, acceptable heat loads can be achieved ($\ll \si{1.MW.m^{-2}}$) for suitably-chosen values of the phase shift. The findings suggest that better confinement can be achieved for $n=3$ and $n=4$ than for $n=2$. This could be due to the fact that the amplitudes of higher $n$ modes fall off more rapidly with distance from the coils, thus reducing the field perturbations in the plasma core where the great majority of high-energy $\alpha$-particles are located.

\begin{figure}[htpb]
    \centering
    \vspace{-1cm}
    \includegraphics[width=0.99\linewidth]{Figures/max_and_total_flux_vs_phase_interior_rmp.pdf}
    \caption{Maximum energy flux (left plots) and percentage heating power lost (right plots) due to deconfined $\alpha$-particles versus ELM suppression coil current phase shift $\Delta\phi$ for different values of $n$ and $I_0$ when in-vessel coils are used. Dashed curves were obtained with vacuum fields, while solid curves show the losses when the plasma response to the perturbations was included. The horizontal dotted line shows results from a simulation in which only the axisymmetric field was used, and the vertical dotted line gives the optimum phase $\Delta\phi_\text{opt}$ for ELM suppression.}
    \vspace{-2cm}
    \label{fig:max_and_total_flux_vs_phase_elm_rwm}
\end{figure}

\begin{figure}[!ht]
    \centering
    \includegraphics[width=0.99\linewidth]{Figures/max_and_total_flux_vs_phase_exterior_rmp.pdf}
    % \caption{Caption}
    % \caption{\parbox{0.9\linewidth}{This figure shows the results from 96 simulations. The left-columns s on the wall in $\si{MW.m^{-2}}$}}
    \caption{As Fig. \ref{fig:max_and_total_flux_vs_phase_elm_rwm} except that ex-vessel coils are employed to suppress ELMs in this case.}
    \label{fig:max_and_total_flux_vs_phase_elm}
\end{figure}

% The optimal $\Delta\phi$ for ELM suppression has been estimated using MARS-F \textbf{cite paper} by applying the criterion that the normalised field perturbation in the plasma should exceed $10^{-4}$: this has been found to be required for effective ELM control in current experiments. In the case of $n = 3$, with the plasma response included, it has been found that ELM suppression is favoured when $\Delta\phi \sim 180^{\circ}$, a value that is somewhat greater than the optimum for minimising $\alpha$-particle losses ($\sim 90^{\circ}$) in this case (see middle row of Fig. \ref{fig:max_and_total_flux_vs_phase_elm}). The choice of $\Delta\phi$ may thus require a compromise between the requirements of ELM suppression and acceptably low $\alpha$-particle losses.

% \subsubsection{Poloidal distribution of RMP-induced heat loads}

% In Fig. \ref{fig:energy_flux_distribution} we show the variation with poloidal angle of heat loads due to $\alpha$-particle losses in one of the more likely RMP scenarios, with $n=3$, $I_0= \si{90.kAt}$, $\Delta \phi = 20^\circ$ and with the plasma response included. More precisely, Fig. \ref{fig:energy_flux_distribution} shows how the $\alpha$-particle energy flux maximised over toroidal angle $S_{max}$ varies with poloidal distance along the first wall, $s_\theta$:
% \begin{equation}
%     \label{eq:max_alpha_particle_energy_flux}
%     S_{max}(s_\theta) = \max_{0\le \phi \le 2\pi}\qty{S(\phi, s_\theta)},
% \end{equation}
% where $S=S(\phi, s_\theta)$ denotes the $\alpha$-particle energy flux on the first wall.
% In the right-hand plot of Fig. \ref{fig:energy_flux_distribution}, the boundaries between the outboard and inboard main chamber walls and the upper and lower divertor regions are marked by vertical lines in blue, orange, green, and red. These are labeled with the symbols +, $\times$, $\bullet$, and $\blacksquare$ respectively. The corresponding symbols and colours are also shown in the left plots for reference. These results show that the largest energy flux is on the dome in the upper divertor, with another significant peak at the end of the outer leg in the lower divertor. As mentioned previously, the maximum tolerable heat loads in these regions are, respectively, around 5 MWm$^{-2}$ and 10 MWm$^{-2}$: the values plotted in Fig. \ref{fig:energy_flux_distribution} are well within these limits. 
% \begin{figure}[htpb]
%     \centering
%     \includegraphics[width=0.99\linewidth]{Figures/energy_flux_distribution_rmp.pdf}
%     \caption{Distribution of peak $\alpha$-particle energy flux maximised over toroidal angle in poloidal cross-section (left) and versus poloidal distance $s_\theta$ along the wall (right plot). Both graphs display the maximum alpha particle energy flux, denoted as $S_{max}$, measured in $\si{MW.m^{-2}}$. The definition of $S_{max}$ can be found in Equation \eqref{eq:max_alpha_particle_energy_flux}. \textbf{I think split this into two rows so we can make both figures signficantly bigger. We need to show DPLOTs of the orbits. Also show this for the full 3D field with RWM and ripple as well.}}
%     \label{fig:energy_flux_distribution}
% \end{figure}

\newpage
\subsubsection{RWM field}
\label{sec:rwm_field}

To maximise fusion power, it is intended that STEP will operate above the no-wall beta limit for $n=1$ resistive wall modes (RWMs). This means that both active feedback and passive control of RWMs will be required \cite{xia2023}. The active feedback coils are activated once the RWM amplitude surpasses a predefined threshold, set at $10^{-4}\,$T in the latest design.
Sensors are located near the active control coils (see Figure \ref{fig:coil_plot_3d}), measuring the field intensity and initiating the feedback cycle in response.

In Fig. \ref{fig:max_and_total_flux_vs_bscale_rwm} we present the results of simulations in which an RWM of fixed amplitude is present. 
We use MARS-F to solve an eigenvalue problem to identify the poloidal structure of the RWM \cite{xia2023}. Since this is a linear calculation, the perturbation amplitude is a free variable.
In \cite{xia2023}, the authors model the RWM and the response of the active feedback coils, including the effects of noise from the sensor signal, in their results. The feedback coils stop the growth of the RWMs, but for any finite noise level there is a residual $n=1$ perturbation in the plasma. The amplitude at the sensor can reach more than ten times the threshold value, meaning that the field there can be as high as $10^{-3}\,$T. In the present paper we only model the RWM and will not include the modification to the field due to the feedback system. We leave the modelling of the full-time-evolving field as a topic for future study. We investigate the scenarios in which the magnitude of the signal detected by the sensors is $10^{-4}\,$T, $10^{-3}\,$T, $10^{-2}\,$T and $10^{-1}\,$T. The results in Fig. \ref{fig:max_and_total_flux_vs_bscale_rwm} indicate that even if the RWM is allowed to reach 100 G at the sensors, the impact on the confinement of the $\alpha$-particles is minimal. If the amplitude of an RWM became so large that the magnitude of the signal detected by the sensors was 100 G or more, it would be likely to cause a disruption and our steady-state model would no longer be applicable. In such a scenario deconfined high energy $\alpha$-particles could pose a threat to plasma facing components in addition to that arising from runaway electrons, but we do not consider such effects here. 

\begin{figure}[!htpb]
    \centering
    \includegraphics[width=0.99\linewidth]{Figures/max_and_total_flux_vs_bscale.pdf}
    \caption{Maximum energy flux (left plot) and percentage heating power lost (right plot) due to deconfined $\alpha$-particles versus RWM amplitude at the sensors located near the RWM control coils. In this case the only non-axisymmetric field is that arising from the RWM. The results in the axisymmetric limit are again shown by dotted horizontal lines.}
    \label{fig:max_and_total_flux_vs_bscale_rwm}
\end{figure}



% Resistive Wall Modes (RWMs) are expected to appear in STEP. To address this, we plan to use RWM coils to generate magnetic fields, allowing active and passive control of RWMs, as described in \cite{xia2023}. Since these RWM fields are three-dimensional, there is a potential for a major effect on the confinement of $\alpha$-particles. Our aim is to thoroughly examine the effects of RWM coils on the $\alpha$-particles and to incorporate these results in our comprehensive submission to the Nuclear Fusion journal. We plan to use 8 RWM coils in each row. Therefore, the sidebands, which were discussed in Section \ref{sec:elm_suppression_field}, will make up a significant part of the RWM field.


\newpage
\clearpage
\subsubsection{All 3D fields}
\label{sec:all_3d_fields}

In this section, we will analyze the $\alpha$-particle losses in a magnetic field composed of the TF ripple field, the field produced by an RWM, and the field generated by the ELM suppression field and subsequent plasma response. This section differs from previous sections, where we examined results from a range of simulations and compared total and peak losses. Here, we will analyse the full distribution of the $\alpha$-particle energy flux on the wall from a single simulation, selecting parameters that closely match the most likely STEP design.

For the TF ripple field, we use $R_{\text{inner}} = 0.75\,\si{m}$ and $R_{\text{outer}} = 9.0\,\si{m}$. We employ out-of-vessel ELM suppression coils with $n=3$, $I_0= \si{90.kAt}$, $\Delta \phi = 20^\circ$, and include the plasma response. For the RWM, we assume it has grown until a 10\,T magnetic field is detected at the detector.

In Fig. \ref{fig:energy_flux_full_3d} we show the variation with poloidal angle of heat loads due to $\alpha$-particle losses. More precisely, Fig. \ref{fig:energy_flux_full_3d} shows how the $\alpha$-particle energy flux maximised over toroidal angle $S_{max}$ varies with poloidal distance along the first wall, $s_\theta$:
\begin{equation}
    \label{eq:max_alpha_particle_energy_flux}
    S_{max}(s_\theta) = \max_{0\le \phi \le 2\pi}\qty{S(\phi, s_\theta)},
\end{equation}
where $S=S(\phi, s_\theta)$ denotes the $\alpha$-particle energy flux on the first wall. Note that the location where we define \( s_\theta = 0 \) is indicated in the top plot of Fig. \ref{fig:energy_flux_full_3d}. The point \( s_\theta = 0 \) is at the midplane on the inboard side of the wall. We define \( s_\theta \) as increasing in the anticlockwise direction around the poloidal cross-section.

In the bottom plot of Fig. \ref{fig:energy_flux_full_3d}, the boundaries between the outboard and inboard main chamber walls and the upper and lower divertor regions are marked by vertical lines in blue, orange, green, and red. These are labeled with the symbols +, $\times$, $\bullet$, and $\blacksquare$ respectively. The corresponding symbols and colours are also shown in the top plot for reference. These results show that the largest energy flux is on the dome in the upper divertor, with another significant peak at the end of the outer leg in the lower divertor. As mentioned previously, the maximum tolerable heat loads in these regions are, respectively, around 5 MWm$^{-2}$ and 10 MWm$^{-2}$: the values plotted in Fig. \ref{fig:energy_flux_full_3d} are well within these limits. 

Figure \ref{fig:energy_flux_full_3d} highlights two major hot spots. The hot spot in the upper dome is labeled as `prompt losses' because most of the markers that impact this region of the wall do so within \(10^{-5}\,\si{s}\) after being generated through nuclear fusion. These markers are generated by $\alpha$-particles born on orbits that quickly take them out of the plasma. The hot spot in the outer leg of the bottom divertor is labeled as `collisional losses' because the $\alpha$-particles take significantly longer to reach this area after their generation. Initially, these particles are well-confined, but they drift due to collisions. It is important to note that the transport time is accelerated by the presence of 3D magnetic fields; this hot spot is not observed in an axisymmetric simulation.

\begin{figure}
    \centering
    \includegraphics[width=0.99\linewidth]{Figures/energy_flux_full_3d.pdf}
    \caption{Distribution of peak $\alpha$-particle energy flux maximised over toroidal angle in poloidal cross-section (left) and versus poloidal distance $s_\theta$ along the wall (right plot). Both graphs display the maximum alpha particle energy flux, denoted as $S_{max}$, measured in $\si{MW.m^{-2}}$. The definition of $S_{max}$ can be found in Equation \eqref{eq:max_alpha_particle_energy_flux}}
    \label{fig:energy_flux_full_3d}
\end{figure}

\newpage
\section{TAE stability calculations}
\label{sec:halo_work}

TAEs are driven unstable due to wave-particle resonances, principally the Landau resonance, which occurs when the particle speed parallel to the magnetic field matches the Alfv\'en speed, $c_A$. The only trans-Alfv\'enic fast ions of any significance in STEP DT plasmas will be the fusion $\alpha$-particles, born with an approximately isotropic velocity distribution clustered around a speed $\sim 1.3\times\si{10^7ms^{-1}}$. This is nearly an order of magnitude higher than the typical values of $c_A$ envisaged in STEP flat-top operation, and therefore the resonance condition will be satisfied by $\alpha$-particles as they slow down. However, although the intrinsic fast ion drive of TAEs is expected to be high in STEP, strong bulk plasma damping of these modes is also expected. The physical reason for this is that STEP will need to be a high plasma $\beta$ device to generate net electrical power, and local values of $\beta$ in the plasma core will be a substantial fraction of unity, meaning that the bulk ion thermal speed $v_i$ will be comparable to $c_A$. Moreover, in addition to the primary Landau resonance $v_{\parallel}=c_A$, in a toroidal plasma there are also sideband resonances $v_{\parallel} = c_A/\vert 2\ell+1 \vert$ where $\ell$ is a positive integer. As a result of these two effects, many bulk ions as well as fast ions can resonate with TAEs in STEP plasmas, and the interaction of bulk ions with these modes is generally expected to result in strong Landau damping.

The STEP equilibrium used in the present study supports a dense spectrum of TAEs. These modes have been identified in the usual way, by first calculating the shear Alfv\'en continuum using the CSCAS code \cite{poedts1993} and then using the MISHKA eigenvalue solver \cite{mikhailovskii1997} to locate TAEs inside the toroidicity-induced continuum gaps. Figure \ref{fig:TAE_structure} shows the low frequency continuum for $n=3$ and the eigenfunction structure of an incompressible mode whose frequency is close to the bottom of the lowest (toroidicity-induced) continuum gap.   
 
\begin{figure}[htpb]
    \centering
    \vspace{-1.7cm}
    \includegraphics[width=1.0\linewidth]{Figures/TAE_figure1.pdf}
    \vspace{-3cm}
    \caption{(a) Thick blue curves: low frequency $n=3$ shear Alfv\'en continua for the equilibrium used in this paper. Frequency $\omega$ is normalised to the Alfv\'en frequency at the magnetic axis $\omega_{A0} = c_A/R_0$ and radial coordinate $s = \psi_N^{1/2}$. Superimposed on the continua are the poloidal harmonics of a TAE with $n=3$ and frequency at the bottom of the continuum gap located at $s \simeq 0.2$. (b) Structure of this mode in the poloidal plane.}
    \label{fig:TAE_structure}
\end{figure}

The stability of these TAEs has been studied using the HAGIS LOCUST (HALO) code \cite{fitzgerald2020}. This is similar to LOCUST in that it is a full-orbit code and utilizes GPU architecture, but it extends LOCUST's capabilities by allowing charged particles to interact resonantly with waves, thereby enabling the calculation of their growth or decay rates. A drift kinetic version of the code has also recently been developed. Intrinsic growth rates (i.e. the $\alpha$-particle drive) of TAEs with a range of $n$ values were calculated using both the drift kinetic and full orbit versions of HALO. The $\alpha$-particles were assumed to have a radial profile and velocity distribution that were consistent with a thermonuclear source corresponding to the temperature and density profiles shown in Fig. \ref{fig:background_plasma_profiles}. The drift kinetic version of HALO has also been used, in separate simulations without $\alpha$-particles, to calculate Landau damping of these modes, assuming Maxwellian velocity distributions and local parameter values again based on those plotted in Fig. \ref{fig:background_plasma_profiles}. These growth and damping rates are plotted in Fig. \ref{fig:TAE_growth_rates}. The key result to note from this figure is that in every case considered the bulk ion Landau damping is sufficient to stabilise the mode. Since other damping processes will be present, for example radiative damping resulting from coupling to kinetic Alfv\'en waves \cite{berk1993}, one may infer that these modes will not be excited in the scenario considered here. Similar calculations have been performed for other STEP flat-top operating points, and it has been found in every case that the bulk ion Landau damping alone exceeds the $\alpha$-particle drive.

\begin{figure}[htpb]
    \centering
    \vspace{-2.5cm}
    \includegraphics[width=1.0\linewidth]{Figures/TAE_figure2.pdf}
    \vspace{-4.0cm}
    \caption{Intrinsic growth rates (blue) and bulk ion Landau damping rates (red) for TAEs with a range of toroidal mode numbers, calculated using HALO (a) in the drift kinetic approximation and (b) with full orbit effects taken into account.}
    \label{fig:TAE_growth_rates}
\end{figure}

The origin of the very strong mode suppression in Fig. \ref{fig:TAE_growth_rates} is apparent from an analytical calculation of the damping rate due to the $\ell = 1$ sideband resonance in the large aspect ratio limit \cite{betti1992}:  
\begin{equation}
    \label{eq:Landau_damping}
    \frac{\gamma_L}{\omega} = -\frac{\pi^{1/2}}{12}q^2\sum_{i=D,T}\beta_i^{1/2}\left[1+\left(1+\frac{2}{9\beta_i}\right)^2\right]\exp\left(-\frac{1}{9\beta_i}\right),
\end{equation}
where $q$ is safety factor, $\beta_i$ is local bulk ion plasma beta, and the summation is over deuterium and tritium. The normalised fast particle drive of TAEs also scales as $q^2$, so the magnitude of damping relative to drive depends critically on $\beta_i$, and the exponential sensitivity of $\gamma_L$ to this parameter means that a relatively modest increase in $\beta_i$ can lead to much stronger damping. A key point to note here is that while the core plasma pressure in STEP is similar to that in other magnetically-confined fusion reactor concepts, the magnetic field is lower, leading to bulk plasma beta values that are high enough to suppress most (possibly all) $\alpha$-particle-driven Alfv\'enic instabilities.   

\section{Conclusions, discussion and future study}
\label{sec:discussion_and_conclusions}
 
As expected, $\alpha$-particles in STEP are well confined in the axisymmetric limit, and TF ripple losses are acceptably low if there are 16 TF coils with outer limbs at major radii of at least $\si{8.m}$. Losses arising from the use of ELM suppression coils pose a more substantial challenge. Our work shows that a suboptimal choice of phase difference $\Delta\phi$ between the upper and lower sets of ELM coils can result in significant deterioration of $\alpha$-particle confinement, and the optimum $\Delta\phi$ for the suppression of ELMs in general differs from that for $\alpha$-particle confinement. However, the losses are within acceptable limits in terms of power load on the wall. More experimental research in spherical tokamaks and in double null devices is needed to establish the requirements for active suppression of ELMs in STEP. However, the work presented here provides useful information to inform the design process, giving us more confidence that a solution to the ELM problem can be found that is compatible with acceptable losses of $\alpha$ particles.

In Section \ref{sec:rwm_field} we studied the losses in a scenario where the magnetic field due to an RWM was included in the model and found that the losses were only slightly higher than the axisymmetric level, even when the perturbation reached a value of $\si{10^{-2}.T}$ at the RWM detector. This magnitude at the detector would be unacceptable, as it could potentially cause a disruption. Therefore, the results suggest that the RWMs, if actively controlled, will not be a problem for $\alpha$-particle losses. However, this analysis did not take into account feedback from the RWM active control, which is a field that varies over time (see \cite{xia2023}) and could significantly modify the poloidal structure of the magnetic field. We intend to investigate the effect of the active control field in the near future, including also the effects of sensor noise and the plasma response to the field generated by the control coils. Additionally, we plan to validate our LOCUST simulation results by using the ASCOT code \cite{hirvijoki2014}. In a similar study on fast ion confinement in the SPARC device, currently under construction, Scott and co-workers \cite{Scott2020} demonstrated that TF coil misalignment can significantly impact fast ion transport. This possibility should also be investigated in STEP.

% To maximise the fusion power, it is intended that STEP will operate above the no-wall beta limit for resistive wall modes (RWMs) with $n=1$. Active and passive control of RWMs will therefore be required \cite{xia2023}. When noise is taken into account, residual field perturbations with dominant $n=1$ are present during RWM control, and these perturbations have the potential to degrade $\alpha$-particle confinement. We therefore plan to model the effects of residual field perturbations due to dedicated RWM coil currents on $\alpha$-particle confinement. The most recent STEP design includes 8 RWM active control coils along the midplane. Toroidal sidebands, discussed in Section \ref{sec:elm_suppression_field}, will make up a significant part of the RWM field and will therefore need to be included in the LOCUST modelling.

Future research in the field of Alfv\'en eigenmode stability in STEP will focus {\it inter alia} on the potential destabilisation of these modes by fast electrons during the ramp-up phase (when electron cyclotron current drive will be employed \cite{Henderson2024}) and by runaway electrons during disruptions. Additionally, the impact of the $q$-profile on Alfv\'en eigenmode stability during the flat-top phase will be explored. We also plan to check the stability of ellipticity-induced Alfv\'en eigenmodes and noncircular triangularity-induced Alfv\'en eigenmodes \cite{betti1992} as well as TAEs. Finally, the stability of higher frequency compressional and global Alfv\'en eigenmodes \cite{Gorelenkov2016} in the presence of fusion $\alpha$-particles will need to be considered. 

% In addition to our focus on the flat-top/steady-state phase, as mentioned in Section \ref{sec:introduction}, it is essential to determine if the $\alpha$-particles are adequately confined during the ramp-up and ramp-down phases.

% % \textbf{For future work we intend to verify our results by collaborating with VTT to check ASCOT gives the same results.}

% % \textbf{Number of ELM suppression coils in each row may be reduced from 16 to 8.}

\section*{Acknowledgements}

This work has been funded by STEP, a major technology and infrastructure programme led by UK Industrial Fusion Solutions Ltd (UKIFS), which aims to deliver the UK’s prototype fusion powerplant and a path to the commercial viability of fusion. The authors thank Antti Snicker and Konsta S\"arkim\"aki at VTT Technical Research Centre of Finland for their help in verifying our results using the ASCOT code.

\section*{Appendix}

\appendix

\section{TF Ripple Analytic Approximation}
\label{appendix:tf_ripple_analytic_approximation}

This appendix demonstrates the accuracy of our equation for the ripple field produced by the TF coils, Equation \eqref{eq:ripple_field}, which we first discussed in Section \ref{sec:tf_ripple_field}. We show that the analytic expression closely matches the numerically calculated ripple field, thus validating its use for efficient computation.

Figure \ref{fig:ripple_check} compares the numerically calculated ripple field with the field given by Equation \eqref{eq:ripple_field} at two vertical positions, \( z = 0 \, \text{m} \) and \( z = 5 \, \text{m} \). The blue curve represents the numerical calculation, and the orange curve shows the analytic one. The close agreement between the two curves indicates that our analytic formula provides an accurate representation of the ripple field. Given this accuracy, we use the analytic formula to save computation time in further simulations. Note that the results in Figure \ref{fig:ripple_check} were calculated with 16 TF coils, \( R_{\text{inner}} = 0.75\, \text{m} \) and \( R_{\text{outer}} = 7.5\, \text{m} \).

\begin{figure}[h!]
    \centering
    \includegraphics[width=0.99\linewidth]{Figures/ripple_check.pdf}
    \caption{Ripple field component \( \delta B_\phi \), normalised by \( B_{axis} \), at \( \phi = 0 \) along \( R \) at \( z = 0 \) (left) and \( z = 5\, \text{m} \) (right). The blue curve was calculated numerically and the orange curve was generated using Equation \eqref{eq:ripple_field}. The vertical dashed lines indicate the \( R \)-coordinates of the last closed flux surface. Note: \(R_\text{inner} = 0.75\, \si{m}\), \( R_{\text{outer}} = 7.5\, \text{m} \), \( N_{\text{coil}} = 16 \).}
    \label{fig:ripple_check}
\end{figure}

In the main text of this paper we focus on how the results change with varying \( R_\text{outer} \), while keeping \( R_\text{inner} \) fixed at \( 0.75\,\text{m} \), the largest plausible \( R_\text{inner} \). Figure \ref{fig:max_and_total_energy_flux_vs_rcoil_inner} shows the maximum $\alpha$-particle energy flux on the walls and the total $\alpha$-power lost for different \( R_\text{inner} \) values. The results show minimal variation for \( R_\text{inner} < 1\,\text{m} \), indicating that the outcomes are not highly sensitive to \( R_\text{inner} \) within this range. The blue curves correspond to \( R_\text{outer} = 7.5\,\text{m} \) and the orange curves correspond to \( R_\text{outer} = 100.0\,\text{m} \).

\begin{figure}[!ht]
    \centering
    \includegraphics[width=0.99\linewidth]{Figures/max_and_total_energy_flux_vs_rcoil_inner.pdf}
    \caption{Peak $\alpha$-particle heat load on the walls (left) and total power lost (right) for a range of \( R_\text{inner} \) values.}
    \label{fig:max_and_total_energy_flux_vs_rcoil_inner}
\end{figure}

\section{Sensitivity of results to background equilibria}
\label{appendix:sensitivity_of_results_to_background_equillibria}

So far we have varied the 3D magnetic field but employed a single axisymmetric background field. Moreover, we have utilized a single plasma scenario: see Figure \ref{fig:background_plasma_profiles} and Table \ref{table:fusion_params}. It is important to stress that the results can be sensitive to changes in the background plasma and magnetic field. With this point in mind, in this appendix we present results obtained for a slightly different plasma scenario, with a subtly different magnetic field and more impurities in the plasma. Some of the key profiles (those of temperature, safety factor, density and fusion reaction rate) are plotted in Figure \ref{fig:plasma_profiles_spr_045_14}. Solid curves represent the the dirtier scenario, while the baseline scenario, which is used in the rest of the paper, is shown as dashed curves. We also present some key quantities of the dirtier scenario in Table \ref{table:fusion_params_dirty}, with the baseline values included for comparison. We use the same set of 3D fields that was employed in Section \ref{sec:all_3d_fields}. The main difference between the two scenarios in terms of impurity content is that argon is present in the dirtier case. This species could be used for the purpose of radiating power in the divertor chambers, thereby moderating the heat flux incident on the divertor plates. Some fraction of the Ar will migrate into the core plasma, resulting in a density profile such as that shown in the bottom left frame of Fig. \ref{fig:plasma_profiles_spr_045_14}. In addition to affecting the fusion reaction rate profile (see bottom right frame of Fig. \ref{fig:plasma_profiles_spr_045_14}), the presence of such impurities impacts on the slowing-down and radial transport of the $\alpha$-particles through collisions.       

In Figure \ref{fig:energy_flux_spr_045_14_vs_spr_045_16}, we show the distribution of $\alpha$-particle energy flux on the wall. The solid blue curve corresponds to the dirtier scenario, and the dashed orange line represents the baseline scenario for reference. We observe that the hotspot in the outer leg of the lower divertor (at approximately \( s_\theta=15\,\si{m} \)) has roughly tripled. This highlights the importance of checking the $\alpha$-particle losses throughout the design process, as small changes in the design can have a large impact on the losses. However, it is worth noting that even though the losses have tripled, they are still well below the tolerable limit, which is about \( 10\,\si{MWm^{-2}} \) in the divertor. Slowed-down $\alpha$-particles in fact are likely to make only a modest contribution to the divertor heat flux. The other main difference in the dirtier scenario is that there is now a small peak in the losses on the inner wall, slightly below the upper divertor ($s_{\theta} \simeq 48\,$m). Again, the absolute fluxes are well below the design limits for this region of the first wall.  

\begin{figure}[!ht]
    \centering
    \includegraphics[width=0.9\linewidth]{Figures/background_plasma_curves_14_vs_16.pdf}
    \caption{Solid curves: profiles of key background plasma parameters for a scenario with higher impurity levels than the baseline, indicated with dashed curves for reference.}
    \label{fig:plasma_profiles_spr_045_14}
\end{figure}

\begin{table}[!ht]
\centering
\begin{tabular}{cccccccc}
\hline
Plasma Scenario & $R_{\text{axis}}$ & $B_{\text{axis}}$ & $I_{\text{p}}$ & $Z_{\text{eff}}$ & $P_{\text{fusion}}$ & $P_{\alpha}$ \\
\hline
Dirtier & 4.21 m & 2.65 T & 20.46 MA & 2.93 & 1.42 GW & 0.28 GW \\
Baseline & 4.38 m & 2.45 T & 20.10 MA & 2.73 & 1.66 GW & 0.33 GW \\
\hline
\end{tabular}
\caption{Key parameters for the dirtier model with higher concentrations of impurities. Baseline values are included for reference, as shown in Table \ref{table:fusion_params}.}
\label{table:fusion_params_dirty}
\end{table}

\begin{figure}[!ht]
    \centering
    \includegraphics[width=0.99\linewidth]{Figures/energy_flux_spr_045_14_vs_spr_045_16.pdf}
    \caption{Solid blue curve: peak $\alpha$-particle energy flux maximised over toroidal angle versus poloidal distance $s_{\theta}$ along the first wall (see Fig. \ref{fig:energy_flux_full_3d}) for the high impurity scenario. Results for the baseline scenario are shown with a dashed orange curve for reference. Note the difference in vertical scales between this figure and Fig. \ref{fig:energy_flux_full_3d}.}
    \label{fig:energy_flux_spr_045_14_vs_spr_045_16}
\end{figure}


\newpage
\section{Kernel density estimation of the heat load on the PFCs}
\label{appendix:kernel_density_estimation}

As stated in Section \ref{sec:model} the LOCUST simulations provide us with the energies and positions of the markers that hit the PFCs. In this appendix we describe how estimates of the energy flux can be obtained from these discrete data using the kernel density estimation (KDE) technique \cite{chen2017}. We could alternatively have used histograms. However, one advantage of KDE is that it limits the number of free parameters. With a histogram, both the size of the bins and the location of the origin are free parameters. For KDE, in contrast, the kernels are centred at the marker locations, making the bandwidth of the kernels the only free parameter.

We parameterize our wall with the coordinates \(s_\theta\) and \(\phi\), where \(s_\theta\) is distance along the wall in the poloidal direction and \(\phi\) is toroidal angle. We estimate the $\alpha$-particle energy flux \(S(\phi, s_\theta)\) on the wall using the following formula:
\begin{equation}
  S(\phi, s_\theta) = P_{\text{lost}}\frac{\sum_{j \in \{\text{set of indices for the markers that hit the wall}\}} K_{h_{\phi}}\qty(\phi - \phi_j; 2\pi) K_{h_{\theta}}(s_\theta - s_{\theta,j}; s_{\theta, \text{max}}) E_{\text{final}, j} w_j}{R \sum_{j \in \{\text{set of indices for the markers that hit the wall}\}} E_{\text{final}, j} w_j},  
\end{equation}
where
\begin{equation}
 K_h(x; L) = \sum_{k=-\infty}^\infty \frac{1}{\sqrt{2\pi} h} \exp\left(-\frac{(x - kL)^2}{2h^2}\right),   
\end{equation}
with \(\phi \in [0, 2\pi]\), \(s_\theta \in [0, s_{\theta, \text{max}}]\), \(\phi_j\) and \(s_{\theta,j}\) denoting the final coordinates of the \(j\)-th marker in the toroidal and poloidal directions, respectively. \(E_{\text{final}, j}\) denotes the final kinetic energy of the \(j\)-th marker, and \(w_j\) is a sampling weight of the \(j\)-th marker. \(K_{h_{\phi}}\) and \(K_{h_{\theta}}\) are the Gaussian kernels we use, with bandwidths \(h_{\phi}\) and \(h_{\theta}\), respectively. These bandwidths are real numbers that need to be calculated.
Note that
\begin{equation}
\int_{s_\theta=0}^{s_{\theta,\text{max}}}\int_{\phi=0}^{2\pi} S(\phi, s_\theta) R \, d\phi \, ds_\theta = P_{\text{lost}},
\end{equation}
as required, provided that $h_{\phi} \ll 2\pi$ and $h_{\theta} \ll s_{\theta,{\rm max}}$.

The bandwidths \(h_{\phi}\) and \(h_{\theta}\) are determined using cross-validation. This involves dividing the data into subsets and systematically training and validating the KDE model on these subsets to optimize the bandwidth parameters. The goal is to select the bandwidths that minimize the error in estimating the true energy flux. Cross-validation ensures that the chosen bandwidths generalize well to unseen data and accurately reflect the underlying distribution.

Finally, we estimate 95\% confidence intervals for the statistical uncertainty using a technique called bootstrap resampling. This technique involves randomly resampling, with replacement, the set of indices for the markers that hit the wall. By resampling with replacement, some indices may be selected multiple times, while others may not be selected at all. For each resampled set, we calculate the energy flux, \(S(\phi, s_\theta)\). At each coordinate, we then determine the range within which 95\% of the calculated energy flux values lie. This range constitutes the 95\% confidence interval for the energy flux at that coordinate.

% Format text for bibliography
% If more than three authors put et al.
\fontsize{9}{12}\selectfont
\setlength{\parskip}{0pt}
\begin{thebibliography}{9}

\bibitem{nuttall2020} % book chapter
% example: CHAPTER-AUTHOR, A., “Title of chapter in sentence case”, Book Title in Title Case, Publisher, Place of Publication (Year).
    WILSON, H., CHAPMAN, I., DENTON, T., et al., 
    ``STEP---on the pathway to fusion commercialization", 
    Commercialising Fusion Energy, 
    IOP Publishing, 
    (2020).

\bibitem{meyer2023} % poster at conference
% example: PRESENTER, A., “Title of presentation in sentence case”, Paper No., paper presented at Organization seminar on subject, Location, year.
    MEYER, H.,
    Plasma burn - mind the gap,
    Phil. Trans. R. Soc. A
    \textbf{382} 
    (2024)
    20230406.

%\bibitem{belova2015} % journal article
% example: AUTHOR, A., AUTHOR, B., AUTHOR, C., Journal article title in sentence case, Abb. J. Title 1 %2 (Year) 120–123.
%    BELOVA, E., GORELENKOV, N., FREDRICKSON, E., et al., 
%    Coupling of neutral-beam-driven compressional Alfv\'en eigenmodes to kinetic Alfv\'en waves in %NSTX tokamak and energy channeling, 
%    Phys. Rev. Lett. 
%    \textbf{115} 1 
%    (2015) 
%    015001.

\bibitem{mitchell2023} % journal article
% example: AUTHOR, A., AUTHOR, B., AUTHOR, C., Journal article title in sentence case, Abb. J. Title 1 2 (Year) 120–123.
    MITCHELL, J., PARROTT, A., CASSON, F., et al.,
    Scenario trajectory optimization and control on STEP,
    Fusion Eng. Des.
    \textbf{192} 
    (2023) 
    113777.

% \bibitem{zsolt2023} % internal report
% % example: AUTHOR, A., Internal Report Title in Title Case, internal report, Organization, Location, Year.
%     ZSOLT, V., et al., 
%     TD-001004, 
%     internal report, 
%     UKAEA, 
%     2023.

\bibitem{garcia-munoz2011} % journal article
% example: AUTHOR, A., AUTHOR, B., AUTHOR, C., Journal article title in sentence case, Abb. J. Title 1 2 (Year) 120–123.
    GARC\'IA-MU\~NOZ, M., CLASSEN, I.G.J., GEIGER, B., et al., 
    Fast ion transport induced by Alfv\'en eigenmodes in the ASDEX Upgrade tokamak, 
    Nucl. Fusion 
    \textbf{51} 
    (2015) 
    103013.

\bibitem{Henderson2024} % journal article
% example: PRESENTER, A., “Title of presentation in sentence case”, Paper No., paper presented at Organization seminar on subject, Location, year.
    FREETHY, S., FIGINI, L., CRAIG, S., et al., 
    The optimisation of the STEP electron cyclotron current drive concept,
    Nucl. Fusion
    \textbf{64} 
    (2024)
    126035.

\bibitem{vaccaro2024} % journal article
% example: AUTHOR, A., Internal Report Title in Title Case, internal report, Organization, Location, Year.
    VACCARO, D., BARRETT, T., BLUTEAU, M., et al.,
    Integrated Methodology for Design and Analysis of First Wall in the STEP Project: A Case Study in the SPR-45 Conceptual Design,
    IEEE Trans. Plasma Sci.,
    \textbf{2024},
    (2024).

\bibitem{bachmann2018} % internal report
% example: AUTHOR, A., Internal Report Title in Title Case, internal report, Organization, Location, Year.
    BACHMANN, C., CIATTAGLIA, S., CISMONDI, F., et al., 
    Overview over DEMO design integration challenges and their impact on component design concepts, 
    Fusion Eng. Des. 
    \textbf{136} 
    (2018)
    87.

%\bibitem{fil2023} % poster at conference
% example: PRESENTER, A., “Title of presentation in sentence case”, Paper No., paper presented at %Organization seminar on subject, Location, year.
%    FIL, A., HENDEN, L., NEWTON, S., et al.,
%    ``Disruption runaway electron generation and mitigation in the Spherical Tokamak for Energy %Production",
%    poster presentation, 
%    29\textsuperscript{th} IAEA Fusion Energy Conference,
%    London, UK, 
%    2023

\bibitem{Berger2022} % journal article
% example: PRESENTER, A., “Title of presentation in sentence case”, Paper No., paper presented at Organization seminar on subject, Location, year.
    BERGER, E., PUSZTAI, I., NEWTON, S., et al.,
    ``Runaway dynamics in reactor-scale spherical tokamak disruptions",
    J. Plasma Phys. 
    \textbf{88} 6 
    (2022)
    905880611.

\bibitem{zohm1996} % journal article
% example: AUTHOR, A., AUTHOR, B., AUTHOR, C., Journal article title in sentence case, Abb. J. Title 1 2 (Year) 120–123.
    ZOHM, H., 
    Edge localized modes (ELMs), 
    Plasma Phys. Control. Fusion 
    \textbf{38} 2 
    (1996) 
    105.

\bibitem{xia2023} % journal article
% example: AUTHOR, A., AUTHOR, B., AUTHOR, C., Journal article title in sentence case, Abb. J. Title 1 2 (Year) 120–123.
    XIA, G., LIU, Y., HENDER, T., et al.,
    Control of resistive wall modes in the spherical tokamak,
    Nucl. Fusion,
    \textbf{63} 2
    (2023)
    026021.

% \bibitem{wagner1982} % journal article
% % example: AUTHOR, A., AUTHOR, B., AUTHOR, C., Journal article title in sentence case, Abb. J. Title 1 2 (Year) 120–123.
%     WAGNER, F., BECKER, G., BEHRINGER, K., et al., 
%     Regime of improved confinement and high beta in neutral-beam-heated divertor discharges of the ASDEX tokamak, 
%     Phys. Rev. Lett. 
%     \textbf{49} 19
%     (1982) 
%     1408.

\bibitem{Evans2008} % journal article
% example: AUTHOR, A., AUTHOR, B., AUTHOR, C., Journal article title in sentence case, Abb. J. Title 1 2 (Year) 120–123.
    EVANS, T., FENSTERMACHER, M., MOYER, R., et al.,
    RMP ELM suppression in DIII-D plasmas with ITER similar shapes and collisionalities,
    Nucl. Fusion,
    \textbf{48} 1 
    (2008) 
    096031.

\bibitem{suttrop2018} % journal article
% example: AUTHOR, A., AUTHOR, B., AUTHOR, C., Journal article title in sentence case, Abb. J. Title 1 2 (Year) 120–123.
    SUTTROP, W., KIRK, A., BOBKOV, V., et al.,
    Experimental conditions to suppress edge localised modes by magnetic perturbations in the ASDEX Upgrade tokamak,
    Nucl. Fusion,
    \textbf{58} 9 
    (2018) 
    096031.

\bibitem{In2019} % journal article
% example: AUTHOR, A., AUTHOR, B., AUTHOR, C., Journal article title in sentence case, Abb. J. Title 1 2 (Year) 120–123.
    IN, Y., LOARTE, A., LEE, H., et al.,
    Test of the ITER-like resonant magnetic perturbation configurations for edge-localized mode crash suppression on KSTAR,
    Nucl. Fusion,
    \textbf{59} 12 
    (2019) 
    096031.
    
\bibitem{van2015} % journal article
% example: AUTHOR, A., AUTHOR, B., AUTHOR, C., Journal article title in sentence case, Abb. J. Title 1 2 (Year) 120–123.
    VAN ZEELAND, M., FERRARO, N., GRIERSON, B., et al.,
    Fast ion transport during applied 3D magnetic perturbations on DIII-D,
    Nucl. Fusion,
    \textbf{55} 7,
    (2015)
    073028.

\bibitem{sanchis2018} % journal article
% example: AUTHOR, A., AUTHOR, B., AUTHOR, C., Journal article title in sentence case, Abb. J. Title 1 2 (Year) 120–123.
    SANCHIS, L., GARCIA-MUNOZ, M., SNICKER, A., et al.,
    Characterisation of the fast-ion edge resonant transport layer induced by 3D perturbative fields in the ASDEX Upgrade tokamak through full orbit simulations,
    Plasma Phys. Control. Fusion,
    \textbf{61} 1,
    (2018)
    014038.
    
\bibitem{ward2022} % journal article
% example: AUTHOR, A., AUTHOR, B., AUTHOR, C., Journal article title in sentence case, Abb. J. Title 1 2 (Year) 120–123.
    WARD, S., AKERS, R., LI, L., et al.,
    LOCUST-GPU predictions of fast-ion transport and power loads due to ELM-control coils in ITER,
    Nucl. Fusion,
    \textbf{62} 12
    (2022)
    126014.

\bibitem{akers2018} % poster at conference
    % example: PRESENTER, A., “Title of presentation in sentence case”, Paper No., paper presented at Organization seminar on subject, Location, year.
    AKERS, R., COLLING, B., HESS, J., et al.,
    ``High fidelity simulations of fast ion power flux driven by 3D field perturbations on ITER",
    poster presentation, 
    26\textsuperscript{th} IAEA Fusion Energy Conference,
    Kyoto, Japan, 
    2016.

\bibitem{ward2021} % journal article
% example: AUTHOR, A., AUTHOR, B., AUTHOR, C., Journal article title in sentence case, Abb. J. Title 1 2 (Year) 120–123.
    WARD, S., AKERS, R., JACOBSEN, A., et al.,
    Verification and validation of the high-performance Lorentz-orbit code for use in stellarators and tokamaks (LOCUST),
    Nucl. Fusion,
    \textbf{61} 8
    (2021)
    086029.

\bibitem{bosch1992} % journal article
% example: AUTHOR, A., AUTHOR, B., AUTHOR, C., Journal article title in sentence case, Abb. J. Title 1 2 (Year) 120–123.
    BOSCH, H., HALE, G.,
    Improved formulas for fusion cross-sections and thermal reactivities,
    Nucl. Fusion,
    \textbf{32} 4
    (1992)
    611-631.

\bibitem{brysk1973} % journal article
% example: AUTHOR, A., AUTHOR, B., AUTHOR, C., Journal article title in sentence case, Abb. J. Title 1 2 (Year) 120–123.
    BRYSK, H.,
    Fusion neutron energies and spectra,
    Plasma Phys.,
    \textbf{15} 7
    (1973)
    611-617.

\bibitem{chen2017} % journal article
% example: AUTHOR, A., AUTHOR, B., AUTHOR, C., Journal article title in sentence case, Abb. J. Title 1 2 (Year) 120–123.
    CHEN, Yen-Chi,
    A tutorial on kernel density estimation and recent advances,
    Biostat. Epidemiol,
    \textbf{1} 1
    (2017)
    161-187.

\bibitem{mcclements2005} % journal article
% example: AUTHOR, A., AUTHOR, B., AUTHOR, C., Journal article title in sentence case, Abb. J. Title 1 2 (Year) 120–123.
    MCCLEMENTS, K.,
    Full orbit computations of ripple-induced fusion $\alpha$-particle losses from burning tokamak plasmas,
    Phys. of Plasmas,
    \textbf{12} 7
    (2005)
    072510.

% \bibitem{ryan2022} % internal report
% % example: AUTHOR, A., Internal Report Title in Title Case, internal report, Organization, Location, Year.
%     RYAN, D., 
%     TD-0014685, 
%     internal report, 
%     UKAEA, 
%     2022.

\bibitem{liu2015} % journal article
% example: AUTHOR, A., AUTHOR, B., AUTHOR, C., Journal article title in sentence case, Abb. J. Title 1 2 (Year) 120–123.
    LIU, Y., AKERS, R., CHAPMAN, I., et al.,
    Modelling toroidal rotation damping in ITER due to external 3D fields,
    Nucl. Fusion,
    \textbf{55} 6
    (2015)
    063027.

\bibitem{ryan2017} % journal article
% example: AUTHOR, A., AUTHOR, B., AUTHOR, C., Journal article title in sentence case, Abb. J. Title 1 2 (Year) 120–123.
    RYAN, D. A., LIU, Y.Q., LI, L., et al.,
    Numerically derived parametrisation of optimal RMP coil phase as a guide to experiments on ASDEX Upgrade,
    Plasma Phys. Control. Fusion,
    \textbf{59} 2 
    (2017) 
    024005.

\bibitem{poedts1993} % journal article
% example: AUTHOR, A., AUTHOR, B., AUTHOR, C., Journal article title in sentence case, Abb. J. Title 1 2 (Year) 120–123.
    POEDTS, S., SCHWARTZ, E.,
    Computation of the ideal-MHD continuous spectrum in axisymmetric plasmas,
    J. Comput. Phys.
    \textbf{105}
    (1993)
    165.

\bibitem{mikhailovskii1997} % journal article
% example: AUTHOR, A., AUTHOR, B., AUTHOR, C., Journal article title in sentence case, Abb. J. Title 1 2 (Year) 120–123.
    MIKHAILOVSKII, A.B., HUYSMANS, G.T.A., KERNER, W.O.K., SHARAPOV, S.E.
    Optimization of computational MHD normal-mode analysis for tokamaks,
    Plasma Phys. Rep.
    \textbf{23}
    (1997)
    844.

\bibitem{fitzgerald2020} % journal article
% example: AUTHOR, A., AUTHOR, B., AUTHOR, C., Journal article title in sentence case, Abb. J. Title 1 2 (Year) 120–123.
    FITZGERALD, M., BUCHANAN, J, AKERS, R.J., et al.,
    HALO: a full-orbit model of nonlinear interaction of fast particles with eigenmodes,
    Computer Phys. Commun.
    \textbf{252}
    (2020)
    106773.

\bibitem{berk1993} % journal article
% example: AUTHOR, A., AUTHOR, B., AUTHOR, C., Journal article title in sentence case, Abb. J. Title 1 2 (Year) 120–123.
    BERK, H.L., METT, R.R., LINDBERG, D.M.,
    Arbitrary mode number boundary-layer theory for nonideal toroidal Alfv\'en modes,
    Phys. Fluids
    \textbf{B5}
    (1993)
    3969.

\bibitem{betti1992} % journal article
% example: AUTHOR, A., AUTHOR, B., AUTHOR, C., Journal article title in sentence case, Abb. J. Title 1 2 (Year) 120–123.
    BETTI, R., FREIDBERG, J.P.,
    Stability of Alfv\'en gap modes in burning plasmas,
    Phys. Fluids
    \textbf{B4}
    (1992)
    1465.

\bibitem{hirvijoki2014} % journal article
% example: AUTHOR, A., AUTHOR, B., AUTHOR, C., Journal article title in sentence case, Abb. J. Title 1 2 (Year) 120–123.
    HIRVIJOKI, E., ASUNTA, O., KOSKELA, T., et al.,
    ASCOT: Solving the kinetic equation of minority particle species in tokamak plasmas,
    Comput. Phys. Commun.,
    \textbf{185} 4 
    (2014) 
    1310-1321.

\bibitem{Scott2020} % journal article
% example: AUTHOR, A., AUTHOR, B., AUTHOR, C., Journal article title in sentence case, Abb. J. Title 1 2 (Year) 120–123.
    SCOTT, S., KRAMER, G., TOLMAN, E., SNICKER, et al.,
    Fast-ion physics in SPARC,
    J. Plasma Phys.,
    \textbf{86}
    (2020) 
    865860508.
\bibitem{Gorelenkov2016} % journal article
% example: AUTHOR, A., AUTHOR, B., AUTHOR, C., Journal article title in sentence case, Abb. J. Title 1 2 (Year) 120–123.
    GORELENKOV, N.,
    Energetic particle-driven compressional Alfv\'en eigenmodes and prospects for ion cyclotron emission studies in fusion plasmas,
    New J. Phys.,
    \textbf{18}
    (2016) 
    105010.

\end{thebibliography}

\end{document}
